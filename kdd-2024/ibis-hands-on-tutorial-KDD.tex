%%
%% This is file `sample-sigconf.tex',
%% generated with the docstrip utility.
%%
%% The original source files were:
%%
%% samples.dtx  (with options: `sigconf')
%% 
%% IMPORTANT NOTICE:
%% 
%% For the copyright see the source file.
%% 
%% Any modified versions of this file must be renamed
%% with new filenames distinct from sample-sigconf.tex.
%% 
%% For distribution of the original source see the terms
%% for copying and modification in the file samples.dtx.
%% 
%% This generated file may be distributed as long as the
%% original source files, as listed above, are part of the
%% same distribution. (The sources need not necessarily be
%% in the same archive or directory.)
%%
%% Commands for TeXCount
%TC:macro \cite [option:text,text]
%TC:macro \citep [option:text,text]
%TC:macro \citet [option:text,text]
%TC:envir table 0 1
%TC:envir table* 0 1
%TC:envir tabular [ignore] word
%TC:envir displaymath 0 word
%TC:envir math 0 word
%TC:envir comment 0 0
%%
%%
%% The first command in your LaTeX source must be the \documentclass command.
\documentclass[sigconf]{acmart}
%% NOTE that a single column version may be required for 
%% submission and peer review. This can be done by changing
%% the \doucmentclass[...]{acmart} in this template to 
%% \documentclass[manuscript,screen]{acmart}
%% 
%% To ensure 100% compatibility, please check the white list of
%% approved LaTeX packages to be used with the Master Article Template at
%% https://www.acm.org/publications/taps/whitelist-of-latex-packages 
%% before creating your document. The white list page provides 
%% information on how to submit additional LaTeX packages for 
%% review and adoption.
%% Fonts used in the template cannot be substituted; margin 
%% adjustments are not allowed.
%%
%%
%% \BibTeX command to typeset BibTeX logo in the docs
\AtBeginDocument{%
  \providecommand\BibTeX{{%
    \normalfont B\kern-0.5em{\scshape i\kern-0.25em b}\kern-0.8em\TeX}}}

%% Rights management information.  This information is sent to you
%% when you complete the rights form.  These commands have SAMPLE
%% values in them; it is your responsibility as an author to replace
%% the commands and values with those provided to you when you
%% complete the rights form.
\setcopyright{acmlicensed}
\copyrightyear{2024}
\acmYear{2024}
\acmDOI{XXXXXXX.XXXXXXX}

%% These commands are for a PROCEEDINGS abstract or paper.
\acmConference[KDD ’24]{30th ACM SIGKDD Conference on Knowledge
Discovery and Data Mining}{August 25--29,2024}{Barcelona, Spain}
% %
% %  Uncomment \acmBooktitle if th title of the proceedings is different
% %  from ``Proceedings of ...''!
% %
\acmBooktitle{KDD '24:ACM SIGKDD Conference on Knowledge
Discovery and Data Mining, August 25--29,2024, Barcelona, Spain} 
% \acmISBN{978-1-4503-XXXX-X/18/06}


%%
%% Submission ID.
%% Use this when submitting an article to a sponsored event. You'll
%% receive a unique submission ID from the organizers
%% of the event, and this ID should be used as the parameter to this command.
%%\acmSubmissionID{123-A56-BU3}

%%
%% For managing citations, it is recommended to use bibliography
%% files in BibTeX format.
%%
%% You can then either use BibTeX with the ACM-Reference-Format style,
%% or BibLaTeX with the acmnumeric or acmauthoryear sytles, that include
%% support for advanced citation of software artefact from the
%% biblatex-software package, also separately available on CTAN.
%%
%% Look at the sample-*-biblatex.tex files for templates showcasing
%% the biblatex styles.
%%

%%
%% The majority of ACM publications use numbered citations and
%% references.  The command \citestyle{authoryear} switches to the
%% "author year" style.
%%
%% If you are preparing content for an event
%% sponsored by ACM SIGGRAPH, you must use the "author year" style of
%% citations and references.
%% Uncommenting
%% the next command will enable that style.
%%\citestyle{acmauthoryear}

%%
%% end of the preamble, start of the body of the document source.
\begin{document}

%%
%% The "title" command has an optional parameter,
%% allowing the author to define a "short title" to be used in page headers.
\title{Ibis: The Data Scientist's Toolkit for Blazing-Fast Analytics, Machine Learning, and Stream Processing}

%%
%% The "author" command and its associated commands are used to define
%% the authors and their affiliations.
%% Of note is the shared affiliation of the first two authors, and the
%% "authornote" and "authornotemark" commands
%% used to denote shared contribution to the research.
\author{Naty Clementi}
\email{add@email.com}
\affiliation{%
  \institution{Voltron Data}
  \state{Virginia}
  \country{USA}
}

\author{Gil Forsyth}
\email{add@email.com}
\affiliation{%
  \institution{Voltron Data}
  \state{New York}
  \country{USA}
}

\author{Chloe He}
\email{add@email.com}
\affiliation{%
  \institution{Voltron Data}
  \state{California}
  \country{USA}
}

%%
%% By default, the full list of authors will be used in the page
%% headers. Often, this list is too long, and will overlap
%% other information printed in the page headers. This command allows
%% the author to define a more concise list
%% of authors' names for this purpose.
% \renewcommand{\shortauthors}{Trovato and Tobin, et al.}

%%
%% The abstract is a short summary of the work to be presented in the
%% article.
\begin{abstract}
Tabular data analysis, a staple in data science, often encounters scalability 
challenges with pandas. In response, modern analytical databases (like DuckDB) 
offer significant performance boosts but typically demand SQL preference and
expertise. Many of these systems only provide a SQL interface though; something
far different from pandas’ dataframe interface, requiring a rewrite of your
analysis code.

This is where Ibis comes in. Ibis is a pure-Python open-source library,
originally created by Wes McKinney (creator of pandas), that provides a
dataframe interface to many popular databases and analytics tools (DuckDB,
Polars, Snowflake, Spark, etc). This lets users analyze data using the same
consistent API, regardless of which backend they’re using, and without ever
having to learn SQL. This tutorial introduces Ibis, emphasizing its role in
simplifying data manipulation across diverse platforms.

Additionally, attendees will explore Ibis' advancements in stream-batch unification. The 
Ibis-flink backend combines stream and batch into a single 
framework, addressing common barriers faced by data scientists venturing into 
streaming analytics. By leveraging Ibis' unified Python dataframe API, we facilitate 
seamless transitions between batch and streaming workflows, reducing operational
complexities and enabling real-time machine learning applications.

Last but not least, this tutorial will also introduce IbisML, an evolving library 
for building machine learning pipelines using Ibis. IbisML brings the portability
of Ibis to ML pipelines -- preprocess your data on any of the 20+ backends that 
Ibis supports, then efficiently handoff to popular training libraries like
scikit-learn, XGBoost, or PyTorch.

Attendees will learn that using Ibis for preprocessing, data transformation, and 
feature engineering allows seamless compilation of pipelines to SQL, enabling 
execution on diverse performant backends without the need for code rewriting 
during production deployment.
\end{abstract}

%%
%% The code below is generated by the tool at http://dl.acm.org/ccs.cfm.
%% Please copy and paste the code instead of the example below.
%%
% \begin{CCSXML}
% <ccs2012>
%  <concept>
%   <concept_id>00000000.0000000.0000000</concept_id>
%   <concept_desc>Do Not Use This Code, Generate the Correct Terms for Your Paper</concept_desc>
%   <concept_significance>500</concept_significance>
%  </concept>
%  <concept>
%   <concept_id>00000000.00000000.00000000</concept_id>
%   <concept_desc>Do Not Use This Code, Generate the Correct Terms for Your Paper</concept_desc>
%   <concept_significance>300</concept_significance>
%  </concept>
%  <concept>
%   <concept_id>00000000.00000000.00000000</concept_id>
%   <concept_desc>Do Not Use This Code, Generate the Correct Terms for Your Paper</concept_desc>
%   <concept_significance>100</concept_significance>
%  </concept>
%  <concept>
%   <concept_id>00000000.00000000.00000000</concept_id>
%   <concept_desc>Do Not Use This Code, Generate the Correct Terms for Your Paper</concept_desc>
%   <concept_significance>100</concept_significance>
%  </concept>
% </ccs2012>
% \end{CCSXML}

% \ccsdesc[500]{Do Not Use This Code~Generate the Correct Terms for Your Paper}
% \ccsdesc[300]{Do Not Use This Code~Generate the Correct Terms for Your Paper}
% \ccsdesc{Do Not Use This Code~Generate the Correct Terms for Your Paper}
% \ccsdesc[100]{Do Not Use This Code~Generate the Correct Terms for Your Paper}

%%
%% Keywords. The author(s) should pick words that accurately describe
%% the work being presented. Separate the keywords with commas.
% \keywords{Do, Not, Us, This, Code, Put, the, Correct, Terms, for,
%   Your, Paper}

%% A "teaser" image appears between the author and affiliation
%% information and the body of the document, and typically spans the
%% page.
% \begin{teaserfigure}
%   \includegraphics[width=\textwidth]{sampleteaser}
%   \caption{Seattle Mariners at Spring Training, 2010.}
%   \Description{Enjoying the baseball game from the third-base
%   seats. Ichiro Suzuki preparing to bat.}
%   \label{fig:teaser}
% \end{teaserfigure}

% \received{20 February 2007}
% \received[revised]{12 March 2009}
% \received[accepted]{5 June 2009}

%%
%% This command processes the author and affiliation and title
%% information and builds the first part of the formatted document.
\maketitle

\section{Target audience and prerequisites}
The target audience for this hands-on tutorial is data scientists, data and machine
learning engineers, researchers, practitioners, and industry professionals who are 
already familiar with data analysis and data manipulation tools in Python. 

They may have encountered scalability challenges with some libraries like pandas 
and are interested in exploring alternative solutions to improve performance and
efficiency.

Additionally, individuals interested in data streaming technology and machine 
learning applications will find value in learning about Ibis' advancements in 
these areas. 

This is a hands-on tutorial, with numerous examples. Participants should ideally
have some experience using Python and dataframe libraries like pandas or polars,
but no SQL experience is necessary.


\section{Tutors biography and expertise}

\textbf{Naty Clementi} is a senior software engineer at Voltron Data, contributing to
Ibis full time. She is a former academic with a Masters in Physics and PhD in
Mechanical and Aerospace Engineering. She is also an active member of Pyladies
and volunteers as one of the directors of Women Who Code in the Washington DC
network. She is originally from Argentina, and speaks both Spanish and English
fluently. Naty is an experienced instructor and educator. She has presented
tutorials at meetups and conferences such as SciPy, PyData NYC, PyData Seattle,
PyCon US. She has also taught multiple Python courses to graduate and
undergraduate engineering students at George Washington University. \\ 

% The most recent Ibis tutorial is the accepted Ibis tutorial at PyCon US 2024: 
% \href{https://us.pycon.org/2024/schedule/presentation/55/}{Tutorials: Introduction to Ibis: blazing fast analytics with DuckDB, Polars, Snowflake, and more, from the comfort of your Python repl.}


\textbf{Gil Forsyth} is a staff software engineer at Voltron Data. He followed the common
career path of Japanese language specialist -> administrative assistant ->
mechanical engineer -> computational fluid dynamicist -> data scientist ->
software engineer -> machine learning engineer -> software engineer. Gil
contributes to several projects in the PyData ecosystem and is a core maintainer
of xonsh and Ibis. Gil is an experienced instructor, having led tutorials at
several PyData conferences, PyCon, and SciPy. He also ran internal training on
Python and distributed data analysis at Capital One for several years. Gil is
one of the core maintainers of Ibis. \\ 


\textbf{Chloe He} has a background in data science and started working on streaming systems
when she was a Founding Engineer at Claypot AI, a startup tackling challenges in
real-time machine learning. She led the infrastructure development of an
open-source real-time feature engineering platform and worked on the translating
and optimizing streaming workloads that served low-latency use cases. Later, she
brought her streaming expertise to Voltron Data, where she now leads the
development of Ibis streaming. Chloe has presented talks and workshops across a number of 
conferences, such as DataConnect, NVIDIA GTC, and Google DevFest. She also taught lectures
for the Machine Learning System Design course at Stanford and a crash course on
recommendation systems. \\

\textbf{Contributors:} Cody Peterson and Deepyaman Datta will help in the preparation of the tutorial. 

\section{Corresponding tutor}

\textbf{Naty Clementi}: add@email.com 

\section{Tutorial Outline}

\begin{enumerate}
\item \textbf{Intro and Setup “Going beyond pandas”}
\vspace{2pt}

Get attendees up and running in a GitHub Codespace or on their laptops. A bit of
motivation about the kinds of problems where Ibis can help, and a general survey 
of attendees to find out what their existing pain points and experiences are.\\

\item \textbf{Introduction to Ibis basics}
\vspace{2pt}
\begin{itemize}
    \item What is Ibis?: Ibis as part of a composable data ecosystem.
    \item Why Ibis?: present the problems that Ibis tackles and the benefits of adopting it.
    \item Who Is Ibis for?: show how data engineers, data analysts, data scientists, and any data practitioner can benefit from Ibis
\end{itemize}
\vspace{2pt}
Get familiarized with basics: A hands-on, follow-along notebook introducing the basic 
verbs of Ibis data analysis (select, filter, group\_by, order\_by, and aggregate), demonstrating 
a simple example on a few local query engines with hands-on exercises throughout.\\

\item \textbf{In-memory tables, joins, and data analysis}
\vspace{2pt}
Building on the previous notebook, we'll explore how to join in-memory data (from 
a pandas DataFrame, polars DataFrame, Python dictionary, or PyArrow Table) with
existing tables in  a local database and continue analysis on the join result.

We'll explore chained joins, and demonstrate \texttt{read\_parquet} and other \texttt{read\_*} methods
for loading local data into existing databases. Then we'll continue with a series 
of hands-on exercises, building up an analysis pipeline for some IMDB ratings data,
but only operating on a 5\% sample of the original dataset. After, we show how the 
same expression can be computed on the full dataset without any code changes, both 
for local execution, or with bursting to a cloud database (or other hosted database).\\

\item \textbf{Streaming}
\vspace{2pt}
We'll build on our previous batch analysis example and turn what we've learned
towards a similar problem, but now with a real-time and constantly updating data
source. 
With hands-on exercises, we'll explore how to set up rudimentary anomaly
detection, and demonstrate how to window and group streaming data-sources before
redirecting them to downstream sinks.\\

\item \textbf{IbisML}
\vspace{2pt}
We will demonstrate how to define machine learning features, create data
transformation pipelines, and execute them across backends using IbisML. We will
then use these features to train an ML model.

\end{enumerate}

\section{Prior Tutorial Comparisons}

\textbf{Ibis: A fast, flexible, and portable tool for data analytics - PyData NYC 2023:} 
This was a short form of an Ibis hands on tutorial, that concentrates on the basics. It was a 90
min tutorial that covered an introduction and hands on example, but did not cover Ibis streaming, or Ibis ML. 

\begin{flushleft}
\textbf{\textit{Github repository}}: \url{https://github.com/ibis-project/ibis-tutorial}
\textbf{\textit{Recording}}: \url{https://www.youtube.com/watch?v=TyopbrmlZx8}
\textbf{\textit{Number attendees}}: ~25-50

\end{flushleft}

\begin{flushleft}
\textbf{Introduction to Ibis: blazing fast analytics with DuckDB, Polars, Snowflake, and more, from the comfort of your Python repl. - PyCon US 2024:} This is a 3h hands on tutorial that is targeted to a broader Python audience,
that will go over Ibis capabilities, covering an introduction and extensive use of the Ibis API through hands-on examples. This tutorial will not showcase streaming and ML capabilities. \\  

\textbf{\textit{Abstract}}: \url{https://us.pycon.org/2024/schedule/presentation/55/}
\textbf{\textit{Note}}: This tutorial will take place on May 2024.\\
\textbf{\textit{Number attendees}}: Expected ~50-100
\end{flushleft}


\section{Participation and Interactivity Strategies}

To encourage audience participation and interactivity throughout the tutorial 
presentation, we will employ several strategies following the general practices 
use by [Software Carpentry](https://software-carpentry.org/), as well as other 
practices that the instructors have found useful. \\

\textbf{Sticky Notes for Status Flags}: Each learner will be provided with two 
sticky notes of different colors, such as red and green. Learners can use these 
to discreetly indicate their status: green for completion or assistance needed, 
and red for encountering problems. This method allows learners to continue 
working while signaling for help, facilitating a smoother learning process for 
everyone. \\

\textbf{Minute Cards for Feedback}: Before each break, learners will take a minute 
to write one positive aspect on a green sticky note and one area for improvement 
or question on a red sticky note. This anonymous feedback allows instructors to 
adjust the pace and content of the tutorial based on learners' needs and interests. \\

\textbf{Pair Programming}: Pairing learners encourages collaborative problem-solving, 
clarification of concepts, and discussion of shared interests. Seating arrangements 
will facilitate pairing, with learners switching partners periodically to maximize
interaction and prevent anyone from feeling isolated. \\

\textbf{Collaborative space to ask questions}: We will use the conference slack, or 
provide a channel for the students to ask questions in a written format while the 
tutorial is being performed. This lets us capture more questions, given that 
sometimes not everyone feels comfortable asking it out loud. \\

\textbf{Peak Rule for Lesson Endings}: We will aim to end each lesson/section on a 
high note, leaving learners with a positive impression of the session. By focusing
on the most intense and final moments, we enhance the overall learning experience
and leave participants feeling motivated and satisfied. \\

\textbf{One Up, One Down Feedback}: At the end of the day, learners will provide
alternating positive and negative feedback points about the day's session. This 
structured feedback encourages participants to express their honest opinions, 
helping instructors to address concerns effectively.


\section{Potential Societal Impacts}

Ibis and therefore this tutorial encapsulate several significant societal impacts, 
particularly in enhancing developer efficiency and optimizing resources.\\

\textbf{Saving Developer Time}: Ibis reduces time spent on database and code migrations 
by providing a consistent API across various backends, allowing developers to 
focus on innovation and product quality. \\

\textbf{Promoting Standardization}: With a unified Python dataframe API, Ibis fosters 
interoperability and collaboration, leading to more efficient workflows and knowledge
sharing. \\

\textbf{Facilitating Access to Advanced Analytics}: By simplifying backend complexities
and offering a familiar Python interface, Ibis lowers barriers for data scientists 
and researchers to leverage advanced analytics techniques and foster innovation. \\


%%
%% The next two lines define the bibliography style to be used, and
%% the bibliography file.
\bibliographystyle{ACM-Reference-Format}
\bibliography{ibis-tutorial}
\nocite{*} % Include all references from the .bib file


\end{document}
\endinput
%%
%% End of file `sample-sigconf.tex'.
